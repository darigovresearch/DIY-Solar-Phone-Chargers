\documentclass{article}
 
\title{DIY Solar: Phone Chargers}
\author{Demand for Energy Equality}
\date{June 2019 \\ Root source version: Version 2.0 \\ Translation version: Version 0.1}

%inherited code
\usepackage{amsmath,amsthm,amssymb,epsfig}
\usepackage{hyperref}

% imports for images
\usepackage{caption}
% \usepackage[demo]{graphicx}
\usepackage{graphicx}
% \usepackage{float}

% import for resuming enumerated list
\usepackage{enumitem}

% code to remove numbering but keep the contents page filled
\setcounter{secnumdepth}{0}

\theoremstyle{definition}
\newtheorem{thm}{Theorem}[section]
\newtheorem{lem}[thm]{Lemma}
\newtheorem{prop}[thm]{Proposition}
\newtheorem{cor}[thm]{Corollary}
\newenvironment{pf}{{\noindent\sc Proof. }}{\qed}

\theoremstyle{definition}
\newtheorem*{defn}{Definition}
\newtheorem*{exmp}{Example}
\newtheorem*{prob}{Problem}
\newtheorem*{info}{Information}
\newtheorem*{warn}{Warning}
\newtheorem*{quest}{Question}
\newtheorem*{blockq}{Block Quote}
\newtheorem*{strong}{Strong}
\newtheorem*{code}{Code}
\newtheorem*{coderesult}{Code Result}

\theoremstyle{remark}
\newtheorem*{rem}{Remark}
\newtheorem*{note}{Note}
\newtheorem*{exer}{Exercise}

\setlength{\oddsidemargin}{0.25 in}
\setlength{\evensidemargin}{-0.25 in}
\setlength{\topmargin}{-0.6 in}
\setlength{\textwidth}{6.5 in}
\setlength{\textheight}{8.5 in}
\setlength{\headsep}{0.75 in}
\setlength{\parindent}{0 in}
\setlength{\parskip}{0.1 in}

%%%%%%%
% Some commonly used notation
%%%%%%%

\def\R{{\mathbb R}}
\def\X{{\mathcal X}}
\def\Y{{\mathcal Y}}
\def\E{{\mathbb E}}
\def\sign{{\rm sign}}

% END inherited code

\begin{document}
 
\maketitle{}

\begin{center}
  \includegraphics[]{../Images/image_0_0_(demand_energy_equality).png}
\end{center}

TODO ADD IMAGE

\vfill
  
\includegraphics[]{../Images/image_0_2_(license).png} \newline
This guide is provided under a \href{https://creativecommons.org/licenses/by-sa/4.0/legalcode}{Creative Commons BY-SA} license: \newline
Material may be freely shared and adapted under the following terms: You must give appropriate credit, provide a link to the license, and indicate if changes were made, and further distribution must be under the same license as the original.

\newpage

\tableofcontents

\newpage

\section{Preface} % (fold)
\label{sec:preface}

  \subsection*{Introduction} % (fold)
  \label{sub:introduction}
  
    This PDF has taken the content of the \href{https://www.demandenergyequality.org/build-your-own-panels}{"DIY Solar: Phone Chargers"} PDF and put them into a form which can be easily corrected, improved and translated by the community using LaTeX a markdown language for technical topics.

  % subsection introduction (end)

  \subsection*{Notes} % (fold)
  \label{sub:notes}

    Please note the modifications which have been made \& where you can find updates.

    \begin{enumerate}
      \item All the content of the PDF and put them into a form which can be easily corrected, improved and translated by the community using LaTeX a markdown language for technical topics.
      \item Any updates, corrections or translations to the PDF will be available at \href{https://github.com/darigovresearch/DIY-Solar-Phone-Chargers}{https://github.com/darigovresearch/DIY-Solar-Phone-Chargers} so do return periodically to check if you have the latest version.
      \item Modifications from the original work includes typo correction, card merging \& consistency consolidation (see the commit history for [en] for the specific changes if any).
    \end{enumerate}

    Feel free to share the PDFs and give the repository a star so more people are likely to see this work and can get the most out of it.

  % subsection notes (end)

  \subsubsection*{License} % (fold)
  \label{ssub:license}

    Unless otherwise specified, everything in this PDF is covered by the following licence:

    \includegraphics[]{../Images/image_0_2_(license).png} \newline

    This work was based on the work \textbf{\textit{DIY Solar: Phone Chargers}} by \href{https://www.demandenergyequality.org/}{Demand Energy Equality}, licensed under a \href{https://creativecommons.org/licenses/by-sa/4.0/legalcode}{Creative Commons BY-SA}.

    To see this work in full go to \href{https://www.demandenergyequality.org/build-your-own-panels}{https://www.demandenergyequality.org/build-your-own-panels}
  
  % subsubsection license (end)

% section preface (end)

\newpage

\section{Introduction} % (fold)
\label{sec:introduction}

  \subsection{The Demand Energy Equality project} % (fold)
  \label{sub:the_demand_energy_equality_project}

    Demand Energy Equality (DEE) is a UK based community energy project that seeks to provide practical energy education using solar photovoltaics. We are a group working for systemic change in the way energy is produced, distributed, controlled, delivered and used. These aims are within the context of rising energy inequality (in the UK, at least), rising fuel bills, climate change and the increasing cost of fossil fuel extraction. See \href{https://www.demandenergyequality.org/about/}{our website} to find out more about the project.

    Through teaching people DIY solar PV skills we also aim to develop their relationship with energy, and enable them to understand it better: where it comes from, how it is used and how it relates to their demand and needs. Ultimately we aim for this to change behaviour, leading to better use of energy and overall reduced demand. Reduced energy use is an unavoidable fact of the relatively near future – far better to prepare now than be surprised later on.

  % subsection the_demand_energy_equality_project (end)
  
  \subsection{Using this guide} % (fold)
  \label{sub:using_this_guide}

    This written guide is for anyone interested in building their own solar phone charger, or learning more about the concepts involved. It assumes no prior knowledge of any kind relevant to building a fully functioning panel. The guide is designed to be used alongside other DIY guides and resources provided by DEE. 

    The guide starts with a summary of the tools and materials used. This is followed by a description of each of the steps in the process of building a portable 10W solar phone charger. At the end of the guide is an appendix with supplemental detailed information about the tools and materials needed. 

    For other types of DIY solar panel that can be made, \href{https://www.instructables.com/}{instructables.com} is a good place to start looking for alternative designs. DEE (and other organisations) run workshops based on other panel designs – you will find information about the \href{https://www.demandenergyequality.org/our-workshops/}{workshops} that we are currently running on the DEE website. 

    You will find the latest version of this guide available to download from the DEE website, as and when this guide is updated, alongside our \href{https://www.demandenergyequality.org/resources/}{other guides and resources}. 

    We encourage you to share the skills you learn with others through your own workshops, particularly if you are able to target and work with low-income communities. Please contact us for any support you feel you may need if you plan to do this.

  % subsection using_this_guide (end)

  \subsection{The design} % (fold)
  \label{ssub:the_design}

    The basic components of the design, and that which makes it economically and practically feasible are the “broken” solar cells. These are cells produced in the industrial manufacture process (mainly in China) that are broken either in transit, or during assembly on arrival. Because they are of no commercial use, these cells can be bought relatively cheaply.

    The solar charger this guide describes is a self-contained design, and can be connected directly to USB devices with no additional equipment needed. 

    This particular guide reflects the latest iteration in the construction of DIY photovoltaic panels as practiced by DEE, but it is likely that it may evolve and expand over time. Because we occasionally introduce new materials and construction methods, the guide may not always be in line with the other DIY resources published by DEE, and may not exactly reflect the content of current workshops. \href{https://www.demandenergyequality.org/contact-us}{Contact DEE} if you need an update on any recent changes. 
  
  % subsection the_design (end)
  
  \subsection{Disclaimer} % (fold)
  \label{sub:disclaimer}

    This guide is for general guidance only and whilst every effort is made to ensure that the information it contains is correct, it should not be relied upon as accurate. The information / advice contained within this guide is intended for use within the UK only and by persons of no less than 18 years of age. Use this guide at your own risk.
    
    DEE will not accept any liability for any loss, damage, injury or negligence direct or indirect from use of the information / advice contained within this guide.

  % subsection disclaimer (end)

  \newpage  

% section introduction (end)

\section{Before starting} % (fold)
\label{sec:before_starting}

  \subsection{Staying safe} % (fold)
  \label{sub:staying_safe}
  
  % subsection staying_safe (end)

  \subsection{Tools and materials} % (fold)
  \label{sub:tools_and_materials}
  
  % subsection tools_and_materials (end)

  \subsection{How to Solder} % (fold)
  \label{sub:how_to_solder}
  
  % subsection how_to_solder (end)

% section before_starting (end)

\section{Building the panel} % (fold)
\label{sec:building_the_panel}

  \subsection{Step 1: Soldering tabbing wire to the top of the cells} % (fold)
  \label{sub:step_1_soldering_tabbing_wire_to_the_top_of_the_cells}
  
  % subsection step_1_soldering_tabbing_wire_to_the_top_of_the_cells (end)

  \subsection{Step 2: Preparing the polycarbonate and placing the cells} % (fold)
  \label{sub:step_2_preparing_the_polycarbonate_and_placing_the_cells}
  
  % subsection step_2_preparing_the_polycarbonate_and_placing_the_cells (end)

  \subsection{Step 3: Heating the EVA to stick the cells} % (fold)
  \label{sub:step_3_heating_the_eva_to_stick_the_cells}
  
  % subsection step_3_heating_the_eva_to_stick_the_cells (end)

  \subsection{Step 4: Tabbing the other side of the cells} % (fold)
  \label{sub:step_4_tabbing_the_other_side_of_the_cells}
  
  % subsection step_4_tabbing_the_other_side_of_the_cells (end)

  \subsection{Step 5: Cross tabbing} % (fold)
  \label{sub:step_5_cross_tabbing}
  
  % subsection step_5_cross_tabbing (end)

  \subsection{Step 6: Encapsulation} % (fold)
  \label{sub:step_6_encapsulation}
  
  % subsection step_6_encapsulation (end)

  \subsection{Step 7: Bonding the panel into the neoprene case} % (fold)
  \label{sub:step_7_bonding_the_panel_into_the_neoprene_case}
  
  % subsection step_7_bonding_the_panel_into_the_neoprene_case (end)

  \subsection{Step 8: Attach USB DC-DC voltage converter} % (fold)
  \label{sub:step_8_attach_usb_dc_dc_voltage_converter}
  
  % subsection step_8_attach_usb_dc_dc_voltage_converter (end)

% section building_the_panel (end)

\newpage

\section{Appendix: Sourcing materials (and possible alternatives)} % (fold)
\label{sec:appendix_sourcing_materials_and_possible_alternatives_}

% section appendix_sourcing_materials_and_possible_alternatives_ (end)

\end{document}